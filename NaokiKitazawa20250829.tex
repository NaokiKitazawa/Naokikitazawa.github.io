\documentclass{article}
\usepackage[whole,autotilde]{bxcjkjatype}
% Thanks to "https://webmail.math.kyushu-u.ac.jp/am_bin/ammain/top?id=31429_806250" and Sasahira-Sensei for preparing a template.
\pagestyle{empty}
\begin{document}
\title{Achievement (研究業績や周辺)}
\author{Naoki Kitazawa}
\date{2025/8/29}
\maketitle
%--------- テンプレート -------------
%
% まず,お名前をお書き下さい.
% なお,この部分はコメントアウトしたままでお願いします.
%
%\par\noindent{北澤\ 直樹(KITAZAWA Naoki)}\\
%
% 注意:名前部分もコメントアウトする.
%    この部分はAnnual report の本体ファイルにコピーするのに
%    使います.  
%
\noindent 
\section{査読あり論文や周辺(以下"〇"が主要論文)。}
\subsection{欧文雑誌に掲載または掲載予定査読あり欧文論文。}

{\bf 1.} (Refereed) N. Kitazawa, \textsl{Fold maps with singular value sets of concentric spheres}, Hokkaido Mathematical Journal Vol.43, No.3 (2014), 327--359, DOI: 10.14492/hokmj/1416837569, \\
https://projecteuclid.org/journals/hokkaido-mathematical-journal/volume-43/issue-3/Fold-maps-with-singular-value-sets-of-concentric-spheres/10.14492/hokmj/1416837569.full.

\par\noindent
〇{\bf 2.} (Refereed) N. Kitazawa, \textsl{Constructions of round fold maps on smooth bundles}, Tokyo J. of Math. Volume 37, Number 2 (2014), 385--403, DOI: 10.3836/tjm/1422452799, https://projecteuclid.org/journals/tokyo-journal-of-mathematics/.%volume-37/issue-2/Constructions-of-Round-Fold-Maps-on-Smooth-Bundles/10.3836/tjm/1422452799.full, arXiv:1305.1708.

\par\noindent
〇{\bf 3.} (Refereed) N. Kitazawa, \textsl{On Reeb graphs induced from smooth functions on $3$-dimensional closed manifolds with finitely many singular values}, Topological Methods in Nonlinear Analysis Vol. 59 No. 2B (2022), 897--912, https://doi.org/10.12775/TMNA.2021.044, arXiv:1902:8841.

\par\noindent
{\bf 4.} (Refereed) N. Kitazawa, \textsl{On Reeb graphs induced from smooth functions on closed or open surfaces}, Methods of Functional Analysis and Topology Vol. 28 No. 2 (2022), 127--143, doi.org/10.31392/MFAT-npu26\_2.2022.05, arXiv:1908.04340.

\par\noindent
〇{\bf 5.} (Refereed) N. Kitazawa, \textsl{Real algebraic functions on closed manifolds whose Reen graphs are given graphs}, Methods of Functional Analysis and Topology Vol. 28 No. 4 (2022), 302--308, arXiv:2302.02339, 2023.

\par\noindent
{\bf 6.} (Refereed) N. Kitazawa and O. Saeki, \textsl{Round fold maps of n--dimensional manifolds into $R^{n-1}$}, J. of Singularities, Vol. 26 (2023), 1--12,\\ http://www.journalofsing.org/volume26/article1.html, DOI: 10.5427/jsing.2023.26a, arXiv:2111.15103.

\par\noindent
{\bf 7.} (Refereed) N. Kitazawa and O. Saeki, \textsl{Round fold maps on $3$-manifolds}, accepted for publication in Algebraic \& Geometric Topology 23 (2023), 3745--3762, arXiv:2105.00974.

\par\noindent
{\bf 8.} (Refereed) N. Kitazawa, \textsl{On Reeb graphs induced from smooth functions on $3$-dimensional closed manifolds which may not be orientable}, Methods of Functional Analysis and Topology Vol. 29 No. 1 (2023), 57--72, 2024.\\
\ \\

\subsection{前述以外の査読有論文や査読有の"プロシーディングの原稿"等。}
\par\noindent
{\bf 1.} (Refereed) N. Kitazawa, \textsl{On round fold maps {\rm (}in Japanese{\rm )}}, RIMS Kokyuroku Bessatsu B38 (2013), 45--59, http://hdl.handle.net/2433/207808.
\par\noindent
{\bf 2.} (Reviewed) N. Kitazawa, \textsl{Explicit construction of explicit real algebraic functions and real algebraic manifolds via Reeb graphs}, Algebraic and geometric methods of analysis 2023 “The book of abstracts”, 49—51, this is the abstract book of the conference "Algebraic and geometric methods of analysis 2023"\\
(https://www.imath.kiev.ua/$\sim$topology/conf/agma2023/) and each abstract is reviewed,\\
https://imath.kiev.ua/$\sim$topology/conf/agma2023/contents/abstracts/texts/kitazawa/kitazawa.pdf. \\
{\bf 3.} (Reviewed) N. Kitazawa, \textsl{On explicit reconstruction of real algebraic maps locally like moment maps}, Algebraic and geometric methods of analysis 2024 “The book of abstracts”, 60—62, this is the abstract book of the conference "Algebraic and geometric methods of analysis 2024"\\
(https://www.imath.kiev.ua/$\sim$topology/conf/agma2024/) and each article is reviewed.\\
{\bf 4.} (Reviewed) N. Kitazawa,  \textsl{Reconstructing Morse functions with prescribed preimages of single points}, Algebraic and geometric methods of analysis 2024 “The book of abstracts”, 60—62, this is the abstract book of the conference "Algebraic and geometric methods of analysis 2025"\\  (https://www.imath.kiev.ua/$\sim$topology/conf/agma2025/agma2025-theses.pdf) and each article is reviewed (https://www.imath.kiev.ua/$\sim$topology/conf/agma2025/index.html\#).
\ \\
\section{学位論文。}
\par\noindent
N. Kitazawa, \textsl{On manifolds admitting fold maps with singular value sets of concentric spheres}, Doctoral Dissertation, Tokyo Institute of Technology (2014).
\section{査読無論文(予稿や報告)。}
\par\noindent
{\bf 1.} 北澤\ 直樹, \textsl{具体的な折り目写像を通したいろいろな多様体の表現}, 研究集会「結び目の数学X」報告集(2018), 25--33.

\par\noindent
{\bf 2.} 北澤\ 直樹, \textsl{折り目写像とその Reeb 空間の位相的情報と定義域多様体}, 研究集会「結び目の数理」報告集(2019), 71--85.

\par\noindent
{\bf  3.} 北澤\ 直樹, \textsl{折り目写像のはめ込み埋め込みや他の折り目写像への持ち上げ{\rm (}可微分写像の特異点論を用いたトポロジー・微分幾何学の研究{)\rm }}, RIMS K\^{o}ky\^{u}roku No. 2140, (2019.12) 23--35.

\par\noindent
{\bf 4.} 北澤\ 直樹, \textsl{Understanding the world of differentiable manifolds via explicit Morse functions and fold maps}, 第 3 回数理新人セミナー報告集 (2020), 130--142, \\ https://drive.google.com/file/d/1fxKa7g5DwGCYKi\_JaTvUNUc0HF1HOXMr/view.

\par\noindent
{\bf 5.} 北澤\ 直樹, \textsl{{\rm Morse}関数折り目写像を介した高次元多様体のより幾何学的構成的な理解}, 「第 17 回数学総合若手研究集会」のシングルセッション用予稿, 「北海道大学数学講究録」180 (2021), 13--22, \\
https://www.math.sci.hokudai.ac.jp/$\sim$ wakate/mcyr/2021/pdf/kitazawa\_naoki.pdf.

\par\noindent
{\bf 6.} 北澤\ 直樹, \textsl{折り目写像の特異性に関する違いと多様体の情報の違いの関係について}, 
研究集会「結び目の数理III」報告集 (2021), 147--156, \\
http://www.math.twcu.ac.jp/mathsciknot3/proc/19Kitazawa.pdf.

\par\noindent
{\bf 7.} 北澤\ 直樹, \textsl{可微分関数の Reeb グラフとグラフの具体的な可微分関数の Reeb グラフとしての実現}, 「第 18 回数学総合若手研究集会」のポスターセッション用予稿, 「北海道大学数学講究録」182 (2022), 789--798. \\
https://www.math.sci.hokudai.ac.jp/$\sim$ wakate/mcyr/2022/pdf/kitazawa\_naoki.pdf.

\par\noindent
{\bf 8.} 北澤\ 直樹, \textsl{Reeb グラフと逆像に関する具体的な条件を満たす滑らかな関数の構成}, 研究集会「結び目の数理VI」 報告集(2024), 173--182, http://www.math.twcu.ac.jp/$\sim$ mathsciknot6/msk6\_proc/21Kitazawa.pdf.

\par\noindent
{\bf 9.} 北澤\ 直樹, \textsl{On realization problems of graphs as Reeb graphs of smooth functions with prescribed preimages}, Singularity theory of smooth maps and its applications, RIMS K\^{o}ky\^{u}roku No. 2226, にて発表予定.
\\
https://www.kurims.kyoto-u.ac.jp/$\sim$ kyodo/kokyuroku/contents/pdf/2226-03.pdf.\\
\ \\


%\medskip
%\noindent{\bf \underline{著書}} \\
%%\noindent なし \\
% \\
\section{プレプリント(主要なもの)。}
{\bf 1.} 
N. Kitazawa, \textsl{Round fold maps and the topologies and the differentiable structures of manifolds admitting explicit ones {\rm (}the title is changed from "On the homeomorphism and diffeomorphism types of manifolds admitting round fold maps"{\rm )}}, submitted to a refereed journal, arXiv:1304.0618, 2019.

\par\noindent
{\bf 2.} N. Kitazawa, \textsl{Maps on manifolds onto graphs locally regarded as a quotient map onto a Reeb space and construction problem}, arXiv:1909.10315.

\par\noindent
{\bf 3.} N. Kitazawa, \textsl{Notes on explicit smooth maps on $7$-dimensional manifolds into the $4$-dimensional Euclidean space}, arXiv:1911.11274. 

\par\noindent
{\bf 4.} N. Kitazawa, \textsl{Explicit fold maps on $7$-dimensional closed and simply-connected manifolds of new classes},  arXiv:2005.5281.

\par\noindent
{\bf 5.} N. Kitazawa, \textsl{7-dimensional simply-connected spin manifolds whose integral cohomology rings are isomorphic to that of $CP^2 \times S^3$ admit round fold maps}, arXiv:2007.03474.

\par\noindent
{\bf 6.} N. Kitazawa, \textsl{Closed manifolds admitting no special generic maps whose codimensions are negative and their cohomology rings}, submitted to a refereed journal, arXiv:2008.04226v5.

\par\noindent
{\bf 7.} N. Kitazawa, \textsl{Notes on explicit special generic maps into Euclidean spaces whose dimensions are greater than $4$}, a revised version is submitted based on positive comments (major revision) by referees and editors after the first submission to a refereed journal, arXiv:2010.10078.

\par\noindent
{\bf 8.} N. Kitazawa, \textsl{Characterizing families of graph manifolds via suitable classes of simple fold maps into the plane and embeddability of Reeb spaces in some 3-dimensional manifolds}, arXiv:2107.08629.

%\par\noindent
%{\bf 9.} N. Kitazawa, \textsl{On Reeb graphs induced from smooth functions on $3$-dimensional closed manifolds which may not be orientable}, this is submitted to a refereed journal based on the second positive comment by editors and referees (major revision), arXiv:2108.01300.

\par\noindent
{\bf 9.} N. Kitazawa, \textsl{
Constructing Morse functions with prescribed preimages of single points
}, submitted to a refereed journal, arXiv:2108.06913
(,where the title has been changed from the title there), 2025.

\par\noindent
{\bf 10.} N. Kitazawa, \textsl{Restrictions on special generic maps on $6$-dimensional or higher dimensional closed and simply-connected manifolds}, arXiv:2201.09437.

\par\noindent
{\bf 11.} N. Kitazawa, \textsl{The $3$-dimensional complex projective space admits no special generic maps}, arXiv:2202.00883.

\par\noindent
{\bf 12.} N. Kitazawa, \textsl{Characterizing certain classes of $6$-dimensional closed and simply-connected manifolds via special generic maps}, arXiv:2205.04048.

\par\noindent
{\bf 13.} N. Kitazawa, \textsl{On the non-existence of special generic maps on complex projective spaces}, arXiv:2206.11500.

\par\noindent
{\bf 14.} N. Kitazawa, \textsl{A class of naturally generalized special generic maps}, submitted to a refereed journal, arXiv:2212.03174.

\par\noindent
{\bf 15.} N. Kitazawa, \textsl{Round fold maps on $3$-dimensional manifolds and their integral and rational cohomology rings}, arXiv:2301.00078.

\par\noindent
{\bf 16.} N. Kitazawa, \textsl{Smooth maps like special generic maps}, arXiv:2301.12126.

\par\noindent
{\bf 17.} N. Kitazawa, \textsl{Restrictions on manifolds admitting certain explicit special-generic-like maps and construction of maps with the manifolds}, arXiv:2302.10079.

\par\noindent
〇{\bf 18.} N. Kitazawa, \textsl{Construction of real algebraic functions with prescribed preimages}, arXiv:2303.00953.

\par\noindent
{\bf 19.} N. Kitazawa, \textsl{Reconstructing real algebraic maps locally like moment-maps with prescribed images and compositions with the canonical projections to the $1$-dimensional real affine space}, the title has changed from previous versions, submitted to a refereed journal, arXiv:2303.10723, 2024.

\par\noindent
{\bf 20.} N. Kitazawa, \textsl{Some remark on real algebraic maps which are topologically special generic maps and generalize the canonical projections of the unit spheres}, submitted to a refereed journal, arXiv:2312.10646, 2024.

\par\noindent
{\bf 21.} N. Kitazawa, \textsl{On a classification of Morse functions on $3$-dimensional manifolds represented as connected sums of manifolds of Heegaard genus one}, arXiv:2411.15943, 2024.

\par\noindent
{\bf 22.} N. Kitazawa, \textsl{Arrangements of small circles for Morse-Bott functions and regions surrounded by them}, arXiv:2412.03846v3, 2024.

\par\noindent
{\bf 23.} N. Kitazawa, \textsl{Characterizing $3$-dimensional manifolds represented as connected sums of Lens spaces, $S^2 \times S^1$, and torus bundles over the circle by certain Morse-Bott functions}, arXiv:2412.11397, 2024.

\par\noindent
{\bf 24.} N. Kitazawa, \textsl{On reconstructing Morse functions with prescribed preimages on $3$-dimensional manifolds and a necessary and sufficient condition for the reconstruction}, submitted to another refereed journal based on the rejection with several positive
comments in the first submission to a refereed journal, a kind of addenda to the article "1.1-8" here, (earlier versions are on)
arXiv:2412.20626, 2025.

\par\noindent
{\bf 25.} N. Kitazawa, \textsl{On reconstructing Morse-Bott functions with prescribed preimages on $3$-dimensional manifolds and conditions for the reconstruction}, after the submission of an earlier version of the previous preprint "24" to a refereed journal the author hit on a new and essential idea extending some result from the preprint to a certain class of Morse-Bott functions from the class of Morse functions explicitly, arXiv:2501.05992, 2025.

\par\noindent
{\bf 26.} N. Kitazawa, \textsl{Arrangements of circles supported by small chords and compatible with natural real algebraic functions}, arXiv:2501.11819, 2025.

\par\noindent
{\bf 27.} N. Kitazawa, \textsl{Realizations of planar graphs as Poincar\'e-Reeb graphs of refined algebraic domains}, arXiv:2501.17425, 2025.

\par\noindent
{\bf 28.} N. Kitazawa, \textsl{Arrangements of circles, the regions surrounded by them and labeled Poincar\'e-Reeb graphs}, arXiv:2502.15195, 2025.

\par\noindent
{\bf 29.} N. Kitazawa, \textsl{Refined algebraic domains with finite sets in the boundaries}, arXiv:2503.09195, 2025.

\par\noindent
{\bf 30.} N. Kitazawa, \textsl{Refined algebraic domains with finite sets in the boundaries respecting differential geometry}, arXiv:2504.04234, 2025.
\par\noindent
{\bf 31.}  N. Kitazawa, \textsl{Moment-like maps and real algebraic functions with prescribed preimages}, arXiv:2506.17791, 2025.

\par\noindent
{\bf 32.} N. Kitazawa, \textsl{Reconstruction of real algebraic functions into curves with prescribed Reeb graphs}, arXiv:2507.09467, 2025.

\par\noindent
{\bf 33.} N. Kitazawa, \textsl{Graphs with tree decompositions of small graphs and realizing them as the Reeb graphs of real algebraic functions}, arXiv:2508.00729, 2025.

\par\noindent
{\bf 34.} N. Kitazawa, \textsl{Planar graphs embedded in generic ways and realizing them as Reeb graphs of real algebraic functions }, arXiv:2508.09498, submitted to https://arxiv.org/, 2025.

\par\noindent
〇{\bf 35.} N. Kitazawa, \textsl{A note on cohomological structures of special generic maps}, a revised version is submitted based on a positive comment by a referee (major revision) after the third submission to a refereed journal, 2024. 
%
%  2018年度の学術講演を古い順に列挙して下さい.また,
%  それ以前の最近5年間程度の学術講演もお書き下さって
%  結構です.
%
% 注意:この間に空行を入れたり\par\noindent
%  を入れたりしないこと.
% 見栄えが悪くなってしまいます.
%
\section{研究集会における講演やポスター発表や世話人等の経験。}
以下 \\
\noindent{\underline{"{\bf International}"とあるものは国際会議での発表とみなせるもの。}\\
	\underline{"{\bf  Invited}" とあるものは"招待されて実施したもの"とみなせるもの。}\\
	\underline{"{\bf Selected}"とあるものは審査を経て講演受理して頂いたもの。}}\\
\subsection{自身の数学研究や周辺に関する講演(2016 年以降)。}

%
%  2018年度の学術講演を古い順に列挙して下さい.また,
%  それ以前の最近5年間程度の学術講演もお書き下さって
%  結構です.
%
% 注意:この間に空行を入れたり\par\noindent
%  を入れたりしないこと.
% 見栄えが悪くなってしまいます.
%
\par\noindent
{\bf 1.} 北澤\ 直樹, 様々な折り目写像のクラスと定義域多様体, 可微分写像の特異点セミナー({\bf 世話人としても参加}), 東京工業大学大岡山キャンパス本館 220, 2016/1/8.

\par\noindent
{\bf 2.} ({\bf  Invited}) 北澤\ 直樹, 基本的な多様体上の折り目写像の構成と関連する話題({\bf 世話人と周辺からの招待講演}), ホモトピー論と特異点論小研究集会, 北海道大学理学部 3 号館 3 - 307 室, 2017/2/15.

\par\noindent
{\bf 3.} 北澤\ 直樹,折り目写像によるいろいろな多様体の表現,結び目の数学 X, 東京女子大学, 2017/12/23.

\par\noindent
{\bf 4.} ({\bf  Invited}) 北澤\ 直樹, special generic 写像とそれに幾何学的に近い折り目写像の幾何学({\bf 世話人と周辺からの招待講演}), 東京学芸大学小研究集会--幾何学と特異点 2018. 2018/3/1.

\par\noindent
{\bf 5.} 北澤\ 直樹, Reeb 空間の最高次元のホモロジー群とサイクルについて, 広島工業大学広島校舎, 微分幾何学・微分式系・特異点論の応用, 2018/6/2.

\par\noindent
{\bf 6.} ({\bf International}) Naoki Kitazawa, \textsl{4-manifolds admitting special generic maps into the 3-dimensional Euclidean space {\rm (}English{\rm )}}, Four Dimensional Topology, Osaka City University, 2018/9/9. 

\par\noindent
{\bf 7.} 北澤\ 直樹, \textsl{可微分写像の Reeb 空間について}, 東北結び目セミナー 2018, カレッジプラザ 講堂, 2018/10/6.

\par\noindent
{\bf 8.} ({\bf International}) Naoki Kitazawa, \textsl{On smooth maps with good geometric properties whose codimensions are negative {\rm (}Poster, English{\rm )}}, Forum "Math-for-Industry" 2018, Fudan University, 2018/11.

\par\noindent
{\bf 9.} ({\bf  Invited}) N. Kitazawa, \textsl{新たな滑らかな写像の持ち上げについて}({\bf 招待講演}), トポロジー金曜セミナー, 九州大学, 2018/11/30.   

\par\noindent
{\bf 10.} 北澤\ 直樹, \textsl{折り目写像の, はめ込み埋め込みや他の折り目写像への持ち上げ}, 可微分写像の特異点論を用いたトポロジー・微分幾何学の研究, 数理解析研究所 110号室. 2018/12/5.

\par\noindent
{\bf 11.}  ({\bf International}) Naoki Kitazawa, \textsl{Constructing smooth maps with good geometric properties whose codimensions are negative {\rm (}Poster, English{\rm )}
}, Ajou-Kyushu joint workshop  on Industrial Mathematics, 九州大学伊都キャンパスウエスト 1 号館 4 階 IMI オーディトリアム, 2018/12.

\par\noindent
{\bf 12.} 北澤\ 直樹, \textsl{折り目写像とその Reeb 空間の位相的情報と定義域多様体}, 研究集会「結び目の数理」, 早稲田大学, 2018/12/24.

\par\noindent
{\bf 13.} ({\bf Selected}) 北澤\ 直樹, \textsl{図形を低次元の空間に写像して調べるという幾何学の手法と錯覚現象等への応用の可能性{\rm (}{\bf 講演申請し採否審査を経ての講演}{\rm )}}, 第 13 回錯覚ワークショップ\ 錯覚現象のモデリングとその応用, 明治大学, 2019/2/25.

\par\noindent
{\bf 14.} 北澤\ 直樹, \textsl{Morse functions and their higher dimensional versions and application to geometry of manifolds and mathematical problems on science and technology}, 数学と諸分野の連携に向けた若手数学者交流会, 科学技術振興機構(JST)東京本部 B1大会議室, 2019/3.

\par\noindent
{\bf 15.} 北澤\ 直樹, \textsl{具体的な折り目写像と定義域多様体}, 2019 年度日本数学会年会トポロジー分科会, 東京工業大学, 2019/3/17.   

\par\noindent
{\bf 16.} 北澤\ 直樹, \textsl{可微分写像の正則値の逆像と Reeb 空間のトポロジー}, 2019 年度日本数学会年会トポロジー分科会, 東京工業大学,  2019/3/17.

\par\noindent
{\bf 17.} ({\bf  Invited}) 北澤\ 直樹, \textsl{Explicit construction of a smooth function whose Reeb graph is a given graph {\rm (}{\bf 招待講演}{\rm )}}, 研究集会「特異点論とトポロジー」({\bf 世話人の一人として参加}), 九州大学,  2019/7/31.

\par\noindent
{\bf 18.} 北澤\ 直樹, \textsl{Smooth functions inducing given graphs as Reeb graphs}, 2019 年度IMI短期共同研究『実践と数理に根ざした多目的最適化ベンチマークの開発』, 九州大学,  2019/9/2.

\par\noindent
{\bf 19.} 北澤\ 直樹, \textsl{与えられたグラフを Reeb グラフとする 3 次元向きづけ可能閉多様体上の具体的な可微分関数の構成}, 2019 年度日本数学会秋季総合分科会, 金沢大学,  2019/9/19.

\par\noindent
{\bf 20.} ({\bf Selected}) 北澤\ 直樹, \textsl{高次元の図形を低次元の空間に射影・写像してみる調べるという幾何学の手法{\rm (}{\bf 講演申請し採否審査を経ての講演}{\rm )}}, 第 12 回関西すうがく徒のつどい, 大阪大学,  2019/10/27.

\par\noindent
{\bf 21.} 北澤\ 直樹, \textsl{Representing various differentiable manifolds via explicit fold maps}, 変換群論シンポジウム, 大阪府立大学 I-site なんば,  2019/10/31.

\par\noindent
{\bf 22.} ({\bf International}) Naoki Kitazawa, \textsl{Fold maps on $7$-dimensional simply-connected closed manifolds {\rm (}Poster, English{\rm )}}, Hyperplane arrangements and Japanese Australian workshop on Real and Complex Singularities, University of Tokyo,  2019/12/5.

\par\noindent
{\bf 23.} 北澤\ 直樹, \textsl{具体的な Morse 関数折り目写像を許容する可微分多様体の位相や可微分構造について}, Poisson geometry and related topics, 立命館大学びわこ・くさつキャンパス,  2019/12/15.

\par\noindent
{\bf 24.} 北澤\ 直樹, \textsl{Understanding the world of differentiable manifolds via  explicit Morse functions and fold maps}, 第 3 回数理新人セミナー, 名古屋大学,  2020/2/10.

\par\noindent
{\bf 25.} 北澤\ 直樹, \textsl{高次元多様体を低次元空間への良い可微分写像を介してみる手法とそこで重要な多面体や組み合わせ的議論について}, 離散数学とその応用研究集会 2020 スペクトルグラフ理論および周辺領域第 9 回研究集会, オンライン,  2020/8/19.

\par\noindent
{\bf 26.} 北澤\ 直樹, \textsl{高次元連結閉多様体のホモロジー群コホモロジー環他幾何的情報の低次元空間への良い可微分写像の具体的構成を介した理解}, 代数、論理、幾何と情報科学研究集会 ALGI31, オンライン, 2020/9/4.


\par\noindent
{\bf 27.} 北澤\ 直樹, \textsl{Understanding algebraic topology and differential topology of higher dimensional closed and simply-connected manifolds in  geometric and constructive ways and related computations}, トポロジーとコンピュータ 2020, オンライン,  2020/9/18.

\par\noindent
{\bf 28.} 北澤\ 直樹, \textsl{$4$ 次元空間への折り目写像の構成で得られる $7$ 次元単連結閉多様体の無限族}, 2020 年度日本数学会秋期総合分科会, 熊本大学(オンライン),  2020/9/24.

\par\noindent
{\bf 29.} 北澤\ 直樹, \textsl{高次元の多様体を低次元の空間への具体的な折り目写像を通してとらえる話とその可能性{\rm (}ポスター{\rm )}}, 日本数学会異分野異業種交流会2020, オンライン,  2020/10/31.

\par\noindent
{\bf 30.} ({\bf  Invited})  北澤\ 直樹, \textsl{具体的な多様体上の具体的な折り目写像と多様体の位相や可微分構造{\rm (}{\bf 「田崎\ 博之\quad  氏{\rm (}筑波大学{\rm )}」による招待講演{\rm )}}}, 筑波大学微分幾何セミナー, オンライン, 2020/11/16.

\par\noindent
{\bf 31.} ({\bf International}) Naoki Kitazawa, \textsl{Understanding the worlds of higher dimensional closed and smooth manifolds of several classes via explicit fold maps {\rm (}Poster, {\bf English}{\rm )}}, 16th Internatinal Workshop on Real \& Complex Singularities {\rm (}online edition{\rm )}, Online, 2020/11/23--30.

\par\noindent
{\bf 32.}  北澤\ 直樹, \textsl{Understanding cohomology rings of closed manifolds via explicit fold maps}, 第 47 回変換群論シンポジウム, オンライン, 2020/12/3.

\par\noindent
{\bf 33.}  北澤\ 直樹, \textsl{折り目写像の特異性に関する違いと多様体の情報の違いの関係について}, 研究集会「結び目の数理III」, オンライン, 2020/12/26.

\par\noindent
{\bf 34.} 北澤\ 直樹, \textsl{高次元単連結閉多様体上の折り目写像について}, 幾何や自然科学に現れる特異点, オンライン, 2021/2/18.

\par\noindent
{\bf 35.} ({\bf Selected}) 北澤\ 直樹, \textsl{{\rm Morse} 関数折り目写像を介した高次元多様体の世界のより幾何学的構成的な理解}, 第 17 回数学総合若手研究集会(講演申請を行い採択率 5--10 %の{\bf 一般の数学者応用数学者向けのシングルセッションでの講演依頼}を受け講演), 北海道大学(オンライン), 2021/3/3.


\par\noindent
{\bf 36.}  ({\bf  Invited}) 北澤\ 直樹, \textsl{Geometric and constructive understanding of the world of higher dimensional manifolds {\rm (}{\bf 「複数の主催者や周辺」による招待講演}{\rm )}}, 数学と諸分野の連携に向けた若手数学者交流会 2 回 2020, 科学技術振興機構 (JST)(オンライン),  2021/3/14 (2020/3/14 に当初予定).

\par\noindent
{\bf 37.} 北澤\ 直樹, \textsl{Special generic 写像と定義域多様体のコホモロジー類の積について}, 2021 年度日本数学会年会トポロジー分科会, 慶應義塾大学(オンライン), 2021/3/16.

\par\noindent
{\bf 38.} 北澤\ 直樹, \textsl{与えられたグラフを Reeb グラフとする閉または開多様体上の具体的な可微分関数の構成}, 2021 年度日本数学会年会トポロジー分科会, 慶應義塾大学(オンライン), 2021/3/16 (口頭発表なしだが規定により講演成立).

\par\noindent
{\bf 39.} ({\bf International}) Naoki Kitazawa,  \textsl{On cohomology classes of manifolds admitting explicit fold maps {\rm (}{\bf English}{\rm )}}, The 27th Osaka City University International Academic Symposium Mathematical Science of Visualization, and Deepening of Symmetry and Moduli, Osaka City University, 2021/3/24 (this was originally scheduled on 2020/3/4--9 as a poster session).

\par\noindent
{\bf 40.}  ({\bf  Invited}) 北澤\ 直樹, Round fold maps on 3-dimensional closed manifolds (「{\bf 古宇田\ 悠哉\quad 氏(広島大学)」による招待講演佐伯修氏との共同研究の内容の発表}), 広島大学トポロジー・幾何セミナー, オンライン, 2021/7/20.

\par\noindent
{\bf 41.} 北澤\ 直樹, 可微分関数・写像と Reeb 空間そしてそれらの具体的な位相的幾何的性質, JCCA-DMIA-2021 離散数学とその応用研究集会 2021, 慶應義塾大学(オンライン), 2021/8/18.

\par\noindent
{\bf 42.} 北澤\ 直樹, Special generic 写像と多様体の Massey 積について, 2021 年度日本数学会年会トポロジー分科会, 千葉大学(オンライン), 2021/9/16.

\par\noindent
{\bf 43.} 北澤\ 直樹, 単連結閉多様体のカップ積と special generic 写像を許容するユークリッド空間の次元, 千葉大学(オンライン), 2021/9/16 (口頭発表なしだが規定により講演成立).

\par\noindent
{\bf 44.} 北澤\ 直樹, グラフ多様体の平面への単純折り目写像を用いた特徴づけについて({\bf 佐伯修氏との共同研究の内容の発表}), 東北結び目セミナー, オンライン, 2021/10/17.

\par\noindent
{\bf 45.} 北澤\ 直樹, On realization problems of graphs as Reeb graphs of smooth functions with prescribed preimages, 京都大学数理解析研究所研究集会「可微分写像の特異点論とその応用」, オンライン, 2021/12/1.

\par\noindent
{\bf 46.} ({\bf International})\ Naoki Kitazawa, Reeb graphs of smooth functions with prescribed preimages ({\bf Poster, English, accepted after a refereeing process by organizing comittee}), FMfI2021, Online, 2021/12/14.

\par\noindent
{\bf 47.}  ({\bf  Invited}) 北澤\ 直樹, グラフ多様体の平面への単純な折り目写像による特徴づけ({\bf 佐伯修氏による依頼講演佐伯修氏との共同研究の内容の発表}), 九大金曜トポロジーセミナー, 2021/12/24.

\par\noindent
{\bf 48.}  Naoki Kitazawa, Reeb graphs of smooth functions with prescribed preimages (ポスター), Pre-Math-for-Innovation Workshop, Online, 2022/1/11.

\par\noindent
{\bf 49.} 北澤\ 直樹, 可微分関数の Reeb グラフとグラフの具体的な可微分関数の Reeb グラフとしての実現(ポスター), 第 18 回数学総合若手研究集会, 北海道大学(オンライン), 2022/3.

\par\noindent
{\bf 50.}  ({\bf  Invited}) 北澤\ 直樹, 与えられたグラフと同型な Reeb グラフを持つような可微分関数の具体的構成について ({\bf 主催者周辺による招待講演}), 多様体と特異点({\bf
	現地世話人手伝いとしても参加}), オンライン(ハイブリッド会場も応募者所属の九州大学に準備), 2022/5.

\par\noindent
{\bf 51.} ({\bf International}) Naoki Kitazawa, \textsl{Round fold maps and construction of ones on some manifolds} ({\bf English}), Singularity theory and its applications, IMS (hybrid: talk via zoom), 2022/10/4.

\par\noindent
{\bf 52.} 北澤\ 直樹, \textsl{Understanding higher dimensional manifolds via special generic maps, their cohomology rings and applications to higher dimensional data}, トポロジーとコンピュータ 2022 , 広島大学, 2022/10/22.

\par\noindent
{\bf 53.} ({\bf International}) Naoki Kitazawa, \textsl{The cohomology rings of the manifolds admitting special generic maps} ({\bf Poster, English}), Deepening and Evolution of Applied Singularity Theory , Workpia Yokohama (Online), 2022/11/25.

\par\noindent
{\bf 54.} ({\bf International}) Naoki Kitazawa,  \textsl{Graph manifolds and round fold maps on them into the plane} ({\bf English}), The 18th East Asian Conference on Geometric Topology, Online, 2023/2/7.

\par\noindent
{\bf 55.} ({\bf International}) Naoki Kitazawa,  \textsl{Explicit construction of explicit real algebraic functions and real algebraic manifolds via Reeb graphs} ({\bf English}), Algebraic and geometric methods of analysis 2023, Online, 2023/5/30.

\par\noindent
{\bf 56.} 北澤\ 直樹,  存在定理を用いない実代数関数・実代数多様体の具体的構成, 特異点論及び周辺分野の深化と異分野への応用, 福岡工業大学, 2023/6/24.

\par\noindent
{\bf 57.} 北澤\ 直樹, Reeb グラフを用いた実代数的関数の具体的構成, JCCA-DMIA-2023 離散数学とその応用研究集会 2023, 愛知教育大学(Hybrid 開催で online にて講演), 2023/8/28.

\par\noindent
{\bf 58.} 北澤\ 直樹,  Round fold maps on $3$-dimensional manifolds and their rational and
 integral cohomology rings, 拡大 KOOK セミナー 2023, 大阪公立大学(Hybrid 開催で online にて講演), 2023/8/30.

\par\noindent
{\bf 59.} 北澤\ 直樹,  「逆像のトポロジー」に関する具体的な条件を満たす滑らかな関数の構成, 第 149 回日本数学会・九州支部例会, 熊本大学, 2023/10/29.

\par\noindent
{\bf 60.} 北澤\ 直樹,  Special generic maps on closed and simply-connected manifolds of dimension $6$, 特異点論の展開, 京都大学数理解析研究所(Hybrid 開催で online にて講演), 2023/11/27.

\par\noindent
{\bf 61.} 北澤\ 直樹,  Reeb グラフと逆像に関する具体的な条件を満たす滑らかな関数の構成, 結び目の数理VI, 東京女子大学, 2023/12/24.

\par\noindent
{\bf 62.} 北澤\ 直樹, グラフを用いた"定義域・大域的構造等がわかる実代数関数"の具体的構成, 第 150 回日本数学会・九州支部例会, 九州工業大学, 2024/2/17. 

\par\noindent
{\bf 63.} ({\bf International}) Naoki Kitazawa, \textsl{On explicit reconstruction of real algebraic maps locally like moment maps ({\bf English})}, Algebraic and geometric methods of analysis 2024, Online, 2024/5/29. 

\par\noindent
{\bf 64.} ({\bf  Invited}) 北澤\ 直樹, 与えられたグラフを Reeb グラフとするような可微分関数・実代数関数の再構成({\bf 「中村\ 伊南沙\ \ 氏(佐賀大学)」による招待講演}), 佐賀創発数理セミナー, 佐賀大学理工学部, 2024/7/12.

\par\noindent
{\bf 65.} 北澤\ 直樹, Constructing real algebraic functions and Reeb graphs of them, トポロジーとコンピュータ 2024, 横浜国立大学, 2024/9/18.

\par\noindent
{\bf 66.}
北澤\ 直樹, Constructing real algebraic functions explicitly and their Reeb graphs, 可微分写像の特異点論とその応用, 京都大学数理解析研究所 420 号室, 2024/12/18. 

\par\noindent
{\bf 67.} ({\bf  Invited, International})
Naoki Kitazawa, \textsl{A classification of Morse functions on $3$-dimensional closed manifolds represented as connected sums of $S^1 \times S^2$ and Lens spaces{\rm (}{\bf English}, 最初に{\bf 佐伯氏より所謂"講演依頼"}があり当初 "{\rm Informal seminar}" とする予定も参加予定者数名で検討し正式に”九大トポロジーセミナー”の”拡大版”とすることに決定{\rm )}}, Extended Kyushu Topology Seminar, 九州大学伊都キャンパス, W1-D725, 2025/1/28.

\par\noindent
{\bf 68.} ({\bf  International})  
Naoki Kitazawa, Reconstructing Morse functions, Morse-Bott functions, or naturally generalized functions with prescribed preimages
({\bf English}), The 20th East Asian Conference on Geometric Topology, 東京大学大学院数理科学研究科, 2025/2/5.

\par\noindent
{\bf 69.}  北澤\ 直樹, Special generic 写像と多様体のコホモロジー, 第 152 回日本数学会九州支部例会(https://www2.math.kyushu-u.ac.jp/$\sim$kyushushibu/PDF/152\_program.pdf: {\bf 当日主催者周辺からのご依頼にて"一般講演 午後の部 I"座長等}), 九州大学伊都キャンパス, 2025/2/15.

\par\noindent
{\bf 70.} ({\bf International}) Naoki Kitazawa, \textsl{Reconstructing Morse functions with prescribed preimages of single points} ({\bf English}: registered in\\ https://imath.kiev.ua/$\sim$topology/conf/agma2025/participants.php),\\ Algebraic and geometric methods of analysis 2025, Online, 2025/5/26.

\par\noindent
{\bf 71.} 北澤\ 直樹, \textsl{Reconstructing Morse functions with prescribed preimages of single points},特異点論と数理科学への応用 , サンフレッシュ山口, 2025/6/5.

\par\noindent
{\bf 72.} Naoki\ Kitazawa, \textsl{Reconstructing Morse functions, Morse-Bott functions, or naturally generalized functions with prescribed preimage}, 力学系の理論と応用, 京都大学数理解析研究所(Hybrid 研究集会にて{\bf オンラインでのみ参加しオンライン講演}), 2025/6/9.

\par\noindent
{\bf 73.}  北澤\ 直樹, \textsl{平面に埋め込まれたグラフ・グラフに自然に縮約する代数的な領域・領域を像とする実代数的写像と定義域の実代数的多様体}, JCCA-DMIA-2025・離散数学とその応用研究集会 2025,
https://sites.google.com/view/jcca-dmia-2025/%E3%83%97%E3%83%AD%E3%82%B0%E3%83%A9%E3%83%A0
, 広島 YMCA 3 号館 3-D, 2025/8/20.
\\
\ \\
\noindent\underline{予定中のもの} \\



\par\noindent
{\bf 74.} 北澤\ 直樹, \textsl{Reconstructing Morse functions with prescribed preimages of single points}, 2025 年度日本数学会秋季総合分科会\\(https://www.mathsoc.jp/assets/file/activity\\
/meeting/nagoya25sept/talklist/talklist25sept\_ja\_20250708.pdf), 名古屋大学, 2025/9/18 (予定).

\par\noindent
{\bf 75.} 北澤\ 直樹, \textsl{Reconstructing Morse functions on $3$-dimensional compact and connected manifolds with prescribed preimages of single points}, 2025 年度日本数学会秋季総合分科会\\(https://www.mathsoc.jp/assets/file/activity\\
/meeting/nagoya25sept/talklist/talklist25sept\_ja\_20250708.pdf), 名古屋大学, 2025/9/18 (予定).

\par\noindent
{\bf 76.} 北澤\ 直樹, \textsl{値域の空間に余次元 $0$ ではめ込まれたコンパクト多様体を像とするような最も自然な special generic 写像と定義域多様体のコホモロジー環},  2025 年度日本数学会秋季総合分科会\\(https://www.mathsoc.jp/assets/file/activity\\
/meeting/nagoya25sept/talklist/talklist25sept\_ja\_20250708.pdf), 名古屋大学, 2025/9/18 (予定).


 \ \\
\noindent\underline{講演申請中のもの} \\
\par\noindent


\par\noindent
{\bf 77.} Naoki Kitazawa, \textsl{Reconstruction of real algebraic maps with prescribed topological conditions and combinatorial ones},  Topology and Computer 2025, 九州大学西新プラザ, 2025/12 (予定).

\ \\
\noindent\underline{中止や延期になったもの} \\
\par\noindent
{\bf 78.}  ({\bf  Invited})  北澤\ 直樹, T. B. A.{\rm (}{\bf 「今田\ 充洋\quad 氏(茨城工業高等専門学校)」による招待講演}{\rm )}, 茨城高専数学セミナー, 茨城工業高等専門学校,  2020/3/23.

\par\noindent
{\bf 79.} 北澤\ 直樹, 領域からの余次元が正でない自然な実代数的写像の再構成と写像の離散数学的情報, JCCA-DMIA-2024 離散数学とその応用研究集会 2024, 山形大学(講演申請もプログラム等の都合で講演不受理), 2024/8. 

\subsection{その他いわゆるアウトリーチ活動など。}

{\bf 80.}
({\bf Invited}) 北澤\ 直樹, 「おいで Math」 ディスカッションパート(招待講演者が数学講演に続き提示された"数学における多様性"に関するテーマを議論するパート)ファシリテーター({\bf 主催者周辺からのご依頼}), Online, 2024/11/7. \\

\par\noindent
{\bf 81.}
({\bf Invited}) 北澤\ 直樹, ようこそマス・フォア・イノベーション連係学府へ, 「マス・フォア・イノベーション連係学府入試説明会」"ヤングメンターからのメッセージ"にて講演({\bf 主催者周辺からのご依頼}), 九州大学伊都キャンパス W1-D413, 2024/12/14.

\ \\
\noindent\underline{予定中のもの} \\
%
% 自由な書式でお書き下さい.
%
%
%
%------------ テンプレート おわり ------------

\end{document}